\documentclass[a4paper,10pt]{article}

\usepackage[utf8]{inputenc}
\usepackage[czech]{babel}
\usepackage[left=2cm,top=3cm,text={17cm,24cm}]{geometry}
\usepackage{graphicx}
\usepackage{listings}
\usepackage{url}
\usepackage{fancyhdr}
\usepackage{titlesec}
\titleformat*{\section}{\fontsize{12}{15}\selectfont}
\titleformat*{\subsection}{\fontsize{11}{13}\selectfont}
\usepackage{courier}
\pagestyle{fancy}
\fancyhf{}
\lhead{Implementační dokumentace k 2. úloze do IPP 2018/2019\\
Jméno a příjmení: Lukáš Drahník\\
Login: xdrahn00}
\rfoot{Page \thepage}

\begin{document}

\hfill

\section*{Návrh programu}
  Při návrhu všech modulů v projektu jsem se snažil dbát na co největší přehlednost kódu, na
  využívání vestavěných knihoven a funkcí a omezit tedy co nejvíce psaní vlastního kód, který snižuje
  čitelnost, zvyšuje šanci na možnou chybu a v neposlední řadě zbytečně prodlužuje samotný zdrojový kód.

  Ve všech modulech je primárně oddělené zpracování argumentů, zpracování vstupu a zápis na výstup.
  Je snaha pro veškerý kód, který na to má nárok vytvořit odpovídající funkci/metodou pro
  znovupoužitelnost a pro zvýšení čitelnosti a rozsekání tak zdrojového kódu na pojmenované sub celky.

  Všechny moduly obsahují nápovědu.

\section{Interpret (\texttt{interpret.py})}
  Interpret je dle zadání psán pro python a verzi 3.6. Vzhledem k návrhu všech částí je i zde funkce na zpracování a validování argumentů (\texttt{parseCmdArgs, validateCmdArgs}) a provedení samotného programu (\texttt{run}).

  Intepretování programu má řadu podpůrných funkcí, stejně jako v \texttt{parse.php} jsou zde funkce na validování předaného XML, používá se zde modul \texttt{xml.etree.ElementTree}, zkráceně \texttt{ET}, dále je zde velký switch pro instrukci, které jsou dále psány jako jednotlivé funkce. U každé funkce se kontroluje počet operandů, typ operandů, zda proměnná existuje a další náležitosti.

  Na regulární výrazy použité především pro kontrolu názvů je použit modul \texttt{re} a metoda \texttt{match}, obdoba php metody \texttt{preg\_match}.

  U instrukce \texttt{BREAK} se vypisuje obsah všech rámců (GF, LF, TF) a hodnoty případného rozšíření \texttt{STATI}.

\section{Vytváření archivu}
  Celý projekt je zabalen do souboru s příponou .tar dle zadání a to pomocí příkazu make tar v souboru Makefile.
  Bylo potřeba se vypořádat s přesunutím dokumentace z podsložky do rootu archivu. Vyřešeno pomocí argumentu \texttt{-C doc readme1.pdf}.

\nocite{*}

%% BIBLIOGRAPHY
\bibliography{local}
\bibliographystyle{plain}

\newpage
\thispagestyle{empty}

\end{document}
%% END OF FILE
